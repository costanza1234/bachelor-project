\documentclass[a4paper]{usiinfbachelorproject}

\captionsetup{labelfont={bf}}
%%%%%%%%%%%%%%%%%%%%%%%%%%%% PACKAGES %%%%%%%%%%%%%%%%%%%%%%%%%%%%%
\usepackage{float}
\usepackage{amsmath}

%%% Main Body %%%

\author{Costanza Rodriguez Gavazzi}

\title{\textbf{How do children search?}}
\subtitle{A tool to support researchers in understanding how children search for information online}
\versiondate{\today}

\begin{committee}
%With more than 1 advisor an error is raised...: only 1 advisor is allowed!
\advisor[Universit\`a della Svizzera Italiana, Switzerland]{ }{Monica }{Landoni }
%You can comment out  these lines if you don't have any assistant
\coadvisor[Universit\`a della Svizzera Italiana, Switzerland]{ }{Diletta Micol}{Tobia}

\end{committee}

\abstract { Abstract goes here ...

Context
Problem
Limitations in SOA
Contribution and Findings

You may include up to six keywords or phrases. Keywords should be separated with semicolons. 
\\
\textbf{Keywords}:

}
\begin{document}
\maketitle
\tableofcontents\newpage
%\listoffigures\newpage

\section{\textbf{Introduction}}

Acknowledge seminal work in the area (very similar tool or research)
In the last paragraph list all your contributions
Add a subsection titled “Report structure” where you briefly discuss
the content of each following section (e.g., In section 2, we review
previous studies in the context of ….).

Context
Problem
Limitations in SOA
Contribution and Findings


Start writing your intro here. You can then use the following commands in your LaTeX document:
\cite{label} To insert a citation where label is the label of a bibliographic entry in a .bib file. For instance:\cite{Hamari}\\




\section{\textbf{Background}}
\subsection{\textbf{subsections}}
Explain all acronyms and abbreviations. For example, the first time an acronym is used, write it out in full and place the acronym in
parentheses. When using the Graphical User Interface (GUI) version, the use may...



\section{\textbf{Approach}}
\subsection{\textbf{The main idea}}
Some of the various techniques are listed below:

\begin{itemize}
    \item tech 1
    \item tech 2
\end{itemize}

\noindent You may write the formulas as follows:

\begin{equation}
    \phi(n) = (p-1) \cdot (q-1)\\
\end{equation}




To insert a figure use the following command:

\begin{figure}[h!]
    \center{\includegraphics[scale=0.03 ]
        {figures/flower.jpg}}
    \caption{The caption of my figure}\label{fig:flower}
\end{figure}




\section{\textbf{Evaluation}}
\subsection{\textbf{Results}}
The experimental result goes here ... \footnote{https://www.usi.ch}. \\







\newpage
\section{\textbf{Future work}}

\section{\textbf{Summary}}
Future works goes here.






\newpage

%%%%% BIBLIOGRAPHY %%%%%
\bibliographystyle{abbrv}
\bibliography{references}

\end{document}
